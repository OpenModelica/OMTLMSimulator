\chapter{BEAST TLM Plugin}

BEAST integrates the TLM plugin directly into the source code. 
Any coordinate system in BEAST has an TLM-enable flag to turn this coordinate system into a TLM interface.

{\scriptsize
\begin{verbatim}
class CtlPoint : public NamedCoordSystem , public MBSTreeComponent {

...

    //! The TLM activation flag.
    int tlmEnabledFlg;

public:

    // Check for tlmEnabled CtlPoint
    bool isTLMEnaled(){ return tlmEnabledFlg != 0; }

...

};
\end{verbatim}
}

The BEAST TLM co-simulation part is based on two C++ classes:
\begin{description}
\item[TLMInterfaceHandler] takes care of creating and initializing all additional components for the TLM co-simulation, this is, a global (cB) coordinate system, necessary connection instances, and the necessary TLMTies.
\item[TLMTie] A TLMTie is created for each TLM interface, that is, for each TLM enabled coordinate system. 
It ties the interface to a global coordinate system for correct motion computation. 
The {\em TLMTie} functions as the communication port between the actual TLM interface in the co-simulation manager.
\end{description}


\section{The TLMInterfaceHandler}
The {\em TLMInterfaceHander} stores a list of all TLM-enabled coordinate systems in the Beast model. 
In the initial phase of the co-simulation it creates all necessary {\em TLMTies} for the communication with the TLM manager. 
This includes creation of a global control point that is needed for the {\em TLMTie} and creation of all necessary {\em cBBodyConnections}.

{\scriptsize
\begin{verbatim}
void TLMInterfaceHandler::EnableTLMCtlPoint(CtlPoint* ctl)
{
    TLMTie* tie=0;
    ModelcBBodyConnection* connection=0;

    // Now, everything seems fine.
    // Let's create the global ctl-point needed for the tie
    if( globalCtl == 0 ){
        assert(topModel!=0);

        doRegister = false;

        globalCtl = new FixedCtlPoint(topModel, "ctTLMglobal" );
        assert(globalCtl!=0);

        doRegister = true;
    }

    // Create the connection if needed
    Connection* tlmConn = 0;
    if( topModel->ConnectionExistQ("cB:"+body->get_FName()) ){
        tlmConn = topModel->GetConnectionPtr("cB:"+body->get_FName());
    }
    else {
        connection = new ModelcBBodyConnection(topModel, body);
        assert(connection!=0);
        tlmConn = connection;
    }

    int idx = tlmConn->Get_ListSize(TIE_category)+1;
    // Now create the TLM tie
    tie = new TLMTie(tlmConn,
                     "TLM" + ToStr(idx),
                     globalCtl,
                     ctl,
		     0, 0,
		     true);
}
\end{verbatim}
}

\section{The TLMTie}

During the co-simulation the {\em TLMTie} is responsible for the communication with the TLM co-simulation manager and takes care of force evaluation and motion propagation in the TLM interface. 
There are three phases. 
First phase is the preparation of the force evaluation. 
This is done in {\em TLMTie::ComputeMasterBefore()}:

{\scriptsize
\begin{verbatim}
// Evaluate the data needed for the current time step.
void TLMTie::ComputeMasterBefore()
{
    if (!NonZeroFlg) return;
    assert(ModelMode != SlaveMode);

    // Get the time data for the specified time
    TLMlink->GetTimeData(ForceID, SimTime, CurTimeData);
}
\end{verbatim}
}

{\em TLMlink->GetTimeData(...)} makes sure that force and moment data for the current time step is available in the interface. 
If it is not yet available it waits until the data has been received though the TLM manager from the connected simulation tool.

Second phase is to evaluate force and moment and update the internal states of the coordinate system. 
This happens in {\em TLMTie::calcChildForceMoment(...)}:

{\scriptsize
\begin{verbatim}
void TLMTie::calcChildForceMoment(const MotionVar& ctl2_M_ctl1_ctl1)
{
    if (!NonZeroFlg) return;

    // Get the motion.
    ctl2_M_ctl1_ctl1.Get(ctl2_R_ctl1_ctl1,
			 ctl2_A_ctl1,
			 ctl2_vR_ctl1_ctl1_ctl1,
			 ctl2_Omega_ctl1_ctl1);

    if(RHSFinalFlg) {
	ctl2_M_ctl1_ctl1_Final =  ctl2_M_ctl1_ctl1;
    }

    double forceOut[6];

    // Note that the static function is used here.
    // This is necessary since the force might be
    // evaluated several times and on a slave.
    TLMPlugin::GetForce(&ctl2_R_ctl1_ctl1(1),
			&ctl2_A_ctl1(1,1),
			&ctl2_vR_ctl1_ctl1_ctl1(1),
			&ctl2_Omega_ctl1_ctl1(1),
			CurTimeData,
			Params,
			forceOut);

    // pc is equal ctl2 for now, i.e. ConLoc12 =1.0
    pc_ctl1_ctl1 = ctl2_R_ctl1_ctl1;;

    double3 F_pc_ctl1_tmp(forceOut[0], forceOut[1],forceOut[2]);
    double3 M_pc_ctl1_tmp(forceOut[3], forceOut[4],forceOut[5]);

    // Transform to system ctl1.
    M2_pc_ctl1 = M_pc_ctl1_tmp;
    M2_ctl1_ctl1 = M_pc_ctl1_tmp + Cross(pc_ctl1_ctl1, F_pc_ctl1_tmp);
    F2_ctl1 = F_pc_ctl1_tmp;


    ctl2_P_ctl1 = F2_ctl1*ctl2_vR_ctl1_ctl1_ctl1 + M2_ctl1_ctl1*ctl2_Omega_ctl1_ctl1;
}
\end{verbatim}
}

Note, that in a parallel BEAST simulation {\em TLMTie::calcChildForceMoment(...)} is invoked on the slaves. 
Slaves do not have a TLMPlugin instance but use the static {\em TLMPlugin::GetForce(...)} instead that requires the TLM parameters as input. 
Time-data and TLM parameters are therefore send to all the slaves using the standard packing mechanism.

The third phase is to send the necessary response to the TLM interface. 
This is done for final/converged solver steps only. 
In BEAST we can check this the {\em RHSFinal} flag:

{\scriptsize
\begin{verbatim}
void TLMTie::ComputeMasterAfter()
    // Set the data - send out the force used in this Evaluate the data needed for the current time step.
{
    if (!NonZeroFlg) return;

    if(RHSFinalFlg) {
	// Get the motion.
	ctl2_M_ctl1_ctl1_Final.Get(ctl2_R_ctl1_ctl1,
				   ctl2_A_ctl1,
				   ctl2_vR_ctl1_ctl1_ctl1,
				   ctl2_Omega_ctl1_ctl1);

	// Set it in TLM Plugin, socket communication might happen
	TLMlink->SetMotion(ForceID,
			   SimTime,
			   &ctl2_R_ctl1_ctl1(1),
			   &ctl2_A_ctl1(1,1),
			   &ctl2_vR_ctl1_ctl1_ctl1(1),
			   &ctl2_Omega_ctl1_ctl1(1));
    }
}
\end{verbatim}
}


