\chapter{MSC.Adams TLM Plugin}

{\bf Important note:}\\ The current version of the co-simulation environment does not propagate delay time (maximum time step) from the composite model to the MSC.ADAMS models automatically. 
Instead these values must be set in the Adams Command File (.acf). 
See Section~\ref{adams:startup-script} for details.


MSC.Adams supports a C subroutine application programming interface (API) that can be integrated with your C/C++ code. 
It it is based on some C header files, for instance:

{
\scriptsize
\begin{verbatim}
// portability header from ADAMS
#include "userPortName.h"

// C-callable subroutines from ADAMS
#include "utilCcallable.h"
\end{verbatim}
}


The MSC.Adams TLMplugin implementation is based on the Adams {\em GFOSUB} subroutine. 
This subroutine can be called from a general force element in Adams.

{
\scriptsize
\begin{verbatim}
// File gfo_wrapper.c
//
// GFOSUB ADAMS/Solver user subroutine interface implementation
//

// GFOSUB - is an interfacec function for general 6-component force tensor
// that is 3-component force & 3-component torque.
//
// id -  id of this external function
// time - current time. Note that adaptive step solver is used in ADAMS
//    Unsuccessful steps are possible. TIMGET function can be used to check
//    is the last call was a final RHS call.
// par - parameters to the function - for GFOSUB expected:
//    a marker, i.e. one parameter
// nPar - number of parameters - should be 1 or 2. If second parameter exists, there will be debug output.
// dflag - true if Jacobian calculation is in progress, i.e. small change of inputs
//      For TLM purposes such calls can be made much faster since TLM force
//       is dependent only on time and not on the states.
// iflag - (init flag) true if solver is only after dependencies between vars
//      We need to make sysary calls to the motion variables that TLM is dependent on.
// result - array of 6 elements with the resulting force
VOID_FUNCTION GFOSUB(int *id, REAL *time, REAL *par, int *nPar, BOOL *dflag, BOOL *iflag, REAL *result){

...

}
\end{verbatim}
}

The {\em c\_sysary} function lets a user-written subroutine read information from ADAMS. 
Here we need it to get the displacement and velocity.

{
\scriptsize
\begin{verbatim}
    //  c_sysary(char *fncnam, int *ipar, int nsize, REAL *states, int *nstate,
    //      BOOL *errflg);
    // Input:
    //  fncnam - name of the function, "DISP" for displacement
    //  ipar, nsize - list of markers and the number of markers
    //    DISP requires 1 to 3 markers where the distance is measured
    //    to marker 1 origin from marker 2 (reference)
    //    in coordinate system of marker 3 (basis) Setting markers
    //     2 and 3 to "0" gives global inertia system as reference.
    //     Markers 2 & 3 are optional.
    //    For our case we use the marker submitted to GFOSUB in parameters
    //    and ground.
    // Output:
    //   states, nstate - variable values and the number of variable returned
    //   For "DISP" 6 variables giving position and orientation are returned
    //   The orientation is given by Psi, Theta, Phi (ADAMS/Solver Euler angles)
    //   c_rcnvrt(char *sys1, REAL *coord1, char *sys2, REAL *coord2, int *istat);
    //   can be used to convert those to Euler parameters by specifying
    //   sys1 = 'EULER' and sys2 = 'EULPAR' or directly to rotation matrix
    //   (columnwise) by specifying sys2 = 'COSINES'.
    //   errflg - returns TRUE on error.

    // Get displacement
    c_sysary ("DISP", markers, 3, disp, &ns, &errflg);

    // Convert the angles from ADAMS "standard" Euler to rotation matrix (9-components)
    c_rcnvrt("EULER",disp+3,"COSINES",rot,&errflg);

    // Get velocity
    c_sysary ("VEL",markers,4,vel,&ns,&errflg);
\end{verbatim}
}

Adams also supports unit scales. 
This is useful since we assume SI units in the TLM interface. 
Thus, we can convert into the correct unit.

{
\scriptsize
\begin{verbatim}
    // Units scaling. See gtunts in ADAMS manual
    BOOL existsUnits;
    double scales[4];
#define UNIT_SCALE_TIME 0
#define UNIT_SCALE_LENGTH 1
#define UNIT_SCALE_FORCE 2
#define UNIT_SCALE_MASS 3
    char units[3 * 4];

    // gtunts retunes the unit scales to MKS as used in the model
    c_gtunts(&existsUnits, scales, units);

    if(existsUnits ) {	
        // Scale transitional measures with length units + speeds with time units
        if(scales[UNIT_SCALE_LENGTH] != 1.0) {
            disp[0] *= scales[UNIT_SCALE_LENGTH];
            disp[1] *= scales[UNIT_SCALE_LENGTH];
            disp[2] *= scales[UNIT_SCALE_LENGTH];

            vel[0] *= scales[UNIT_SCALE_LENGTH];
            vel[1] *= scales[UNIT_SCALE_LENGTH];
            vel[2] *= scales[UNIT_SCALE_LENGTH];	
        }
        if(scales[UNIT_SCALE_TIME] != 1.0) {
            for( i = 0; i < 6; ++i) {
               vel[i] /= scales[UNIT_SCALE_TIME];
            }
        }
    }
\end{verbatim}
}

Finally we can call {\em calc\_tlm\_force(...)} that invokes both, {\em SetMotion(...)} and {\em GetForce(...)} in the TLM interface. 
In order to handle ``final'' or ``converged'' solver steps we invoke the Adams function {\em c\_timget(...)} that returns the last converged time step. 
This information is passed on to the TLM interface for the correct TLM parameter communication with the connected simulation tools, that is, data is send for converged steps only.

{
\scriptsize
\begin{verbatim}
    // TIMGET returns simulation time at the end of the last successful step
    c_timget(&lastConvergedTime);

    calc_tlm_force(*iflag,  // init flag. No result is needed.
                   *dflag,  // If set, it is derivatives calculation.
                            // Time stands still.
                   dbgOut,     // should we print debug messages?
                   markers[0], // The calling marker ID
                   *time,      // Current simulation time
                   lastConvergedTime, // Last converged time
                   disp,    // Marker position data
                   rot,     // Marker rotation matrix
                   vel,     // Marker translational velocity
                   vel + 3, // Marker angular velocity
                   result);  // Output 6-component force
\end{verbatim}
}

The calculated load (force and moment) response needs to be converted back to Adams units.

{
\scriptsize
\begin{verbatim}
    if(existsUnits ) {
        // Scale force & torque
        if(scales[UNIT_SCALE_FORCE] != 1.0) {
            for( i = 0; i < 6; ++i) {
                result[i] /= scales[UNIT_SCALE_FORCE];
            }
            if(scales[UNIT_SCALE_LENGTH] != 1.0) {
                result[3] /= scales[UNIT_SCALE_LENGTH];
                result[4] /= scales[UNIT_SCALE_LENGTH];
                result[5] /= scales[UNIT_SCALE_LENGTH];
            }
        }
    }
};
\end{verbatim}
}


In the ADAMS TLM plugin implementation we keep track of the last converged time step in order to send data only if needed, that is, when we have a converged step. 
This is implemented in the {\em TLM\_force::GetForce(...)} function that is invoked by {\em calc\_tlm\_force(...)}, see File~{\em tlmforce.c} in the TLMPlugin/ADAMS directory.

What happens is that we always store the motion data of the last time step that ADAMS invoked {\em calc\_tlm\_force(...)} for. 
For each step we check if the last step was a converged step (based on the information we got from the ADAMS solver and the data that we stored).
If it was a converged step we send the data that we have cached, that is, the last time step.

{
\scriptsize
\begin{verbatim}
void TLM_force::GetForce(bool derCalc,
                         int markerID,
                         double lastConvergedTime,
                         MarkerMotionData& param,
                         double* force) {
    if(getMode() == 0) {
        // init has been called for all
        setMode();
        SwitchToRunMode();
    }

    MarkerID& mID = MarkerIDmap[markerID];

    int interfaceID = mID.ID; // interface force ID in TLM manager

    MarkerMotionData& lastParam = LastMarkerMotion[mID.index];

    if(!derCalc) { // if it's a normal call (not Jacobian)
        if( (lastParam.Time >= 0 ) // there's data
            && (lastParam.Time != param.Time ) //not a repeated call
            ) {
	
            if(lastConvergedTime == lastParam.Time) { // that was a converged step
                map<int, MarkerID>::iterator it;
                for(it = MarkerIDmap.begin(); it != MarkerIDmap.end();++it) {
                    int curID = it->second.ID;
                    int index = it->second.index;
                    MarkerMotionData& toSend = LastMarkerMotion[index];
                    Plugin->SetMotion(curID,          // Send data to the Plugin
                                  toSend.Time,
                                  toSend.Position,
                                  toSend.Orientation,
                                  toSend.Speed,
                                  toSend.Ang_speed);
                    // invalidate time to avoid resend
                    toSend.Time = param.Time;
                }
            }
       }

       lastParam = param; // store the current motion data
    }

    // Call the plugin to get reaction force
    Plugin->GetForce(interfaceID,
                     param.Time,
                     param.Position,
                     param.Orientation,
                     param.Speed,
                     param.Ang_speed,
                     force);
}

\end{verbatim}
}



