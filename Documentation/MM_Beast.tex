\chapter{External BEAST Models}
The TLM-Plugin has been implemented for BEAST. This section describes
how to integrate BEAST models into Meta-Models and Meta-Model
simulations.

\section{TLM Enabled BEAST models}
A BEAST model needs to be ``TLM enabled'' to participate in a
Meta-Model simulation. The TLM-plugin has been integrated into the
BEAST code and is thus always available in BEAST models.

Control-points are the interfaces for TLM connections in BEAST. One
can enable any fixed and flexible control-point in BEAST for TLM
connections. For TLM connections it is, however, recommended to use
control-points that are not connected to a tie.

Any BEAST model that has at least one TLM enabled control-point can
participate in a Meta-Model (co-)simulation. To enable a control-point
for TLM communication one needs to set the {\em TLMEnabledFlg} for
this control point in the following way:
\begin{itemize}
\item Right-click on the control point in the model-browser
\item From the pop-up menu select {\em Edit Variables...} and then
{\em TLM}
\item In the dialog set the {\em TLMEnabledFlg} to {\em Yes}.
\end{itemize}


The BEAST model can then be saved and integrated into as an external
model into a Meta-Model.


\section{BEAST Startup Script}
A startup script for running the BEAST simulation needs to be created
as well. A default script is distributed with the BEAST/MME
installation. It is called StartTLMBeast.bat.

{\bf Note:} The default \verb!StartTLMBeast.bat! Windows script needs to
be checked. This should probably be done together with a system
administrator and a member of the BEAST team. However below is a short
discussion of the script for advanced users.

Figure \ref{figStartBeast} presents a template for the start script.
The \verb!StartTLMBeast.bat! should first generate a file \verb!<CaseName>.tlm!
that will contain the parameter send to it by TLM manager.
Only the line giving start command for BEAST needs to be changed.

\begin{figure}[h]
\small{
\begin{verbatim}
@echo off
set BeastCmd="%BEAST%/bin/%ABI%/FORMAT-9.3/Beast_Serial"

echo execution directory is %1
cd %1

echo Starting a Beast simulation with input file: %6
echo Make sure that:
echo time = %2
echo timeEnd = %3
echo MaxTimeStep "<"= %4
echo Writing caseID %1 and server name %5 to file %6.tlm
echo %1 > %6.tlm
echo %5 >> %6.tlm
echo %2 >> %6.tlm
echo %3 >> %6.tlm
echo %4 >> %6.tlm

echo Starting beast
echo %BeastCmd% %6.in

%BeastCmd% %6.in > %6.simlog
\end{verbatim}
}
\caption{A template for the \emph{startadams.bat} script\label{figStartBeast}}
\end{figure}

