\chapter{Simulink TLM Plugin}

The Matlab/Simulink TLMPlugin implementation is based on a Matlab
S-function interface.

{\scriptsize
\begin{verbatim}
/*
 * tlmforce.c: Based on 'C' template for a level 2 S-function.
 *
 *  -------------------------------------------------------------------------
 *  | See matlabroot/simulink/src/sfuntmpl_doc.c for a more detailed template |
 *  -------------------------------------------------------------------------
 *
 * Copyright 1990-2000 The MathWorks, Inc.
 * $Revision$
 */
\end{verbatim}
}

TLM delay is set in the initialization of the S-Function interface
function {\em mdlInitializeSampleTimes(...).}

{\scriptsize
\begin{verbatim}
/* Function: mdlInitializeSampleTimes =========================================
 * Abstract:
 *    This function is used to specify the sample time(s) for your
 *    S-function. You must register the same number of sample times as
 *    specified in ssSetNumSampleTimes.
 */
static void mdlInitializeSampleTimes(SimStruct *S)
{
    double sTime, eTime, timeStep;
    TLM_InterfaceReg::GetInstance(false)->GetSimParameters(sTime, eTime, timeStep);
    // true or false in GetInstance(...) enables/disables debug output

    TLMErrorLog::Log("Set sample time to " + ToStr(timeStep));

    /* Set TLM delay here! */
    ssSetSampleTime(S, 0, CONTINUOUS_SAMPLE_TIME);
    ssSetOffsetTime(S, 0, 0.0);

    /* Set TLM delay here! */
    ssSetSampleTime(S, 1, timeStep);
    ssSetOffsetTime(S, 1, 0.0);
}
\end{verbatim}
}

Note, that the first call to {\em TLM\_InterfaceReg::GetInstance(...)}
initialized the TLM plugin, that is, it reads the TLM configuration
parameters form the config file.

Communication with Simulink is based on the C structure {\em
  SimStruct}. Initially we need to setup the size (of the
data that we want to communicate in the structure. This is done in the
S-Function interface function {\em mdlInitializeSizes(...)}.

{\scriptsize
\begin{verbatim}
/* Function: mdlInitializeSizes ===============================================
 * Abstract:
 *    The sizes information is used by Simulink to determine the S-function
 *    block's characteristics (number of inputs, outputs, states, etc.).
 */
static void mdlInitializeSizes(SimStruct *S)
{
    /* See sfuntmpl_doc.c for more details on the macros below */

    ...

    /* input ports are: */
    /* position[3]      Interface position data */
    /* orientation[3x3] Interface rotation matrix */
    /* speed[3]         Interface translational velocity */
    /* ang_speed[3]     Interface angular velocity */

    if (!ssSetNumInputPorts(S, 4)) return;
    /* Set size of input arrays */
    ssSetInputPortWidth(S, 0, 3);
    ssSetInputPortWidth(S, 1, 9);
    ssSetInputPortWidth(S, 2, 3);
    ssSetInputPortWidth(S, 3, 3);

    ...

    /* output ports are: */
    /* force[3]  Output force */
    /* moment[3] Output moment */

    if (!ssSetNumOutputPorts(S, 4)) return;
    ssSetOutputPortWidth(S, 0, 3);
    ssSetOutputPortWidth(S, 1, 3);
    ssSetOutputPortWidth(S, 2, 3);
    ssSetOutputPortWidth(S, 3, 9);

}
\end{verbatim}
}

The force calculation in the TLM interface takes place in the
S-Function interface function {\em mdlOutputs(...)}. This function is
called by Simuink for a given time instance. The motion of the
interface can be extracted from the {\em SimStruct} structure that has
been setup before, see {\em mdlInitializeSizes} above. The resulting
load (force and moment) is stored in the same structure.

{\scriptsize
\begin{verbatim}
/* Function: mdlOutputs =======================================================
 * Abstract:
 *    In this function, you compute the outputs of your S-function
 *    block. Generally outputs are placed in the output vector, ssGetY(S).
 */
static void mdlOutputs(SimStruct *S, int_T tid)
{
    /* inputs: */
    /* position[3]      Interface position data */
    /* orientation[3x3] Interface rotation matrix */
    /* speed[3]         Interface translational velocity */
    /* ang_speed[3]     Interface angular velocity */
    double *R = (double*)ssGetInputPortSignal(S,0);
    double *A = (double*)ssGetInputPortSignal(S,1);
    double *vR = (double*)ssGetInputPortSignal(S,2);
    double *Omega = (double*)ssGetInputPortSignal(S,3);

    const char* name = ssGetPath(S);
    real_T time = ssGetT(S);

    int ifID = TLM_InterfaceReg::GetInstance()->GetInterfaceID(name);

    /* output */
    double force[6];
    TLMTimeData CurTimeData;

    if( ifID >= 0 ) {
        TLM_InterfaceReg::GetInstance()->GetPlugin()->GetForce(ifID,
                                                               time,
                                                               R,
                                                               A,
                                                               vR,
                                                               Omega,
                                                               force);

        /* Get Position and Orientation */
        TLM_InterfaceReg::GetInstance()->GetPlugin()->GetTimeData(ifID,
                                                                  time,
                                                                  CurTimeData);

    }
    else {
        /* Not connected */
        for( int i=0 ; i<6 ; i++ ) {
            force[i] = 0.0;
        }
    }

    /* ------- store the result ------- */

    /* Force & Moment */
    real_T  *f = ssGetOutputPortRealSignal(S,0);
    real_T  *m = ssGetOutputPortRealSignal(S,1);

    for( int i=0 ; i<3 ; i++ ){
        f[i] = force[i];
        m[i] = force[i+3];
    }

    /* Position & Orientation */
    real_T  *R_TLM = ssGetOutputPortRealSignal(S,2);
    real_T  *A_TLM = ssGetOutputPortRealSignal(S,3);

    for( int i=0 ; i<3 ; i++ ){
        R_TLM[i] = CurTimeData.Position[i];
        A_TLM[i] = CurTimeData.RotMatrix[i];
        A_TLM[i+3] = CurTimeData.RotMatrix[i+3];
        A_TLM[i+6] = CurTimeData.RotMatrix[i+6];
    }

}
\end{verbatim}
}

{\em TLM::SetMotion(...)} is called when we want to update the states
for a ``final'' integration step. In the S-Function interface this is
handled in the function {\em mdlUpdate(...)}. This triggers the
communication with the connected simulation tools in order to
propagate the states.

{\scriptsize
\begin{verbatim}
  /* Function: mdlUpdate ======================================================
   * Abstract:
   *    This function is called once for every major integration time step.
   *    Discrete states are typically updated here, but this function is useful
   *    for performing any tasks that should only take place once per
   *    integration step.
   */
  static void mdlUpdate(SimStruct *S, int_T tid)
  {
      if( ssIsSampleHit(S, 1, tid) ){

          double *R = (double*)ssGetInputPortSignal(S,0);
          double *A = (double*)ssGetInputPortSignal(S,1);
          double *vR = (double*)ssGetInputPortSignal(S,2);
          double *Omega = (double*)ssGetInputPortSignal(S,3);

          const char* name = ssGetPath(S);
          real_T time = ssGetT(S);

          int ifID = TLM_InterfaceReg::GetInstance()->GetInterfaceID(name);

          if( ifID >= 0 ){
            // Send data to the Plugin
            TLM_InterfaceReg::GetInstance()->GetPlugin()->SetMotion(ifID,
                                                                    time,
                                                                    R,
                                                                    A,
                                                                    vR,
                                                                    Omega);
          }
      }
  }
\end{verbatim}
} 